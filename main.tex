\documentclass[journal,12pt,twocolumn]{IEEEtran}
\usepackage{tikz}
\usepackage{amsmath}
\usepackage{breqn}
\usepackage{amssymb}
\pagestyle{empty}
\usepackage{setspace}
\usepackage{gensymb}
\singlespacing
\usepackage{mathtools} 
\usepackage{amsmath}
\usepackage{amsthm}
\begin{document}
\providecommand{\sbrak}[1]{\ensuremath{{}\left[#1\right]}}
\providecommand{\lsbrak}[1]{\ensuremath{{}\left[#1\right.}}
\providecommand{\rsbrak}[1]{\ensuremath{{}\left.#1\right]}}
\providecommand{\brak}[1]{\ensuremath{\left(#1\right)}}
\providecommand{\lbrak}[1]{\ensuremath{\left(#1\right.}}
\providecommand{\rbrak}[1]{\ensuremath{\left.#1\right)}}
\providecommand{\cbrak}[1]{\ensuremath{\left\{#1\right\}}}
\providecommand{\lcbrak}[1]{\ensuremath{\left\{#1\right.}}
\providecommand{\rcbrak}[1]{\ensuremath{\left.#1\right\}}}
\newcommand{\myvec}[1]{\ensuremath{\begin{pmatrix}#1\end{pmatrix}}}
\newcommand{\cmyvec}[1]{\ensuremath{\begin{pmatrix*}[c]#1\end{pmatrix*}}}
\newcommand{\mydet}[1]{\ensuremath{\begin{vmatrix}#1\end{vmatrix}}}
\newcommand{\proj}[2]{\textbf{proj}_{\vec{#1}}\vec{#2}}
\let\StandardTheFigure\thefigure
\let\vec\mathbf

\title{
Assignment - 1
}
\author{ Nisarg Parekh \\SM21MTECH14002}
\maketitle
\newpage
\bigskip
\bibliographystyle{IEEEtran}
\section*{\textbf{Problem}}
\noindent
\textbf{\textsl{1.Show that the area of the triangle whose vertices are:
$$(a \tan\theta, b\cot\theta),(a \tan \phi, b\cot \phi),(a \tan \gamma, b\cot \gamma)~~$$
$$ is~~  \frac{4ab~\sin(\theta-\phi)\sin(\phi-\gamma)\sin(\gamma-\theta)}{\sin2\theta\sin2\phi\sin2\gamma}$$ }}
\noindent
\section*{\textbf{Solution}}
\noindent
\begin{align*}
 \vec{x_1} = \mydet{a\tan\theta}, \vec{x_2} = \myvec{a\tan\phi}
\vec{x_3}=\myvec{a\tan\gamma}\\\vec{y_1}=\myvec{b\cot\theta},\vec{y_2}=\myvec{b\cot\phi},\vec{y_3}=\myvec{b\cot\gamma}\end{align*}
\begin{multline}
area~ of ~triangle=~\frac{1}{2} \mydet{
 x_{1} & y_{1} & 1  \\ 
 x_{2} & y_{2} & 1  \\
 x_{3} & y_{3} & 1 
}\end{multline}

\begin{multline}
=~\frac{1}{2}~\mydet{
 a\tan\theta & b\cot\theta & 1  \\ 
 a\tan\phi & b\cot\phi & 1  \\
 a\tan\gamma & b\cot\gamma & 1 
}\end{multline}\\
\begin{multline}
=~\frac{ab}{2}~\mydet{
 \tan\theta & \cot\theta & 1  \\ 
 \tan\phi &   \cot\phi & 1  \\
 \tan\gamma & \cot\gamma & 1 
}
\end{multline}
$\because$~taking~ common ~$ C1 \xleftrightarrow{{a}} C1$  ~and~ $C2 \xleftrightarrow{{b}} C2$

\begin{multline}
=~\frac{ab}{2}~\mydet{
 \tan\theta & \cot\theta & 1  \\ 
 \tan\phi-\tan\theta & \cot\phi-\cot\theta & 0  \\
 \tan\gamma-\tan\theta & \cot\gamma-\cot\theta & 0 
}\end{multline}
$\because R2 \xleftrightarrow{{R2-R1}} R2$  ~and~ $R3 \xleftrightarrow{{R3-R1}} R3$


\begin{multline}
=~\frac{ab}{2}~\mydet{
 \frac{\sin\theta}{\cos\theta} & \frac{\cos\theta}{\sin\theta} & 1  \\ 
 \frac{\sin\phi}{\cos\phi}-\frac{\sin\theta}{\cos\theta} & \frac{\cos\phi}{\sin\phi}-\frac{\cos\theta}{\sin\theta} & 0  \\
 \frac{\sin\gamma}{\cos\gamma}-\frac{\sin\theta}{\cos\theta} & \frac{\cos\gamma}{\sin\gamma}-\frac{\cos\theta}{\sin\theta} & 0 
}\end{multline}
\begin{align*}
\because \tan x = \frac{\sin x}{\cos x}~and~\cot x = \frac{\cos x}{\sin x} \end{align*}

\begin{multline}
=\frac{ab}{2}~\mydet{
 \frac{\sin\theta}{\cos\theta} & \frac{\cos\theta}{\sin\theta} & 1  \\ 
 \frac{\sin\phi\cos\theta - \sin\theta\cos\phi}{\cos\phi\cos\theta} & \frac{\sin\theta\cos\phi - \sin\phi\cos\theta}{\sin\phi\sin\theta} & 0  \\
  \frac{\sin\gamma\cos\theta - \sin\theta\cos\gamma}{\cos\gamma\cos\theta} & \frac{\sin\theta\cos\gamma - \sin\gamma\cos\theta}{\sin\gamma\sin\theta} & 0
}   
\end{multline}

$\because$ ~rearranging ~terms 


\begin{multline}
=\frac{ab}{2}~\mydet{
 \frac{\sin\theta}{\cos\theta} & \frac{\cos\theta}{\sin\theta} & 1  \\ 
 \frac{\sin(\phi-\theta)}{\cos\phi\cos\theta} & \frac{\sin(\theta-\phi)}{\sin\phi\sin\theta} & 0  \\
 \frac{\sin(\gamma-\theta)}{\cos\gamma\cos\theta} & \frac{\sin(\theta-\gamma)}{\sin\gamma\sin\theta} & 0 
}    
\end{multline}
\begin{align*}
\because \sin A\cos B - \sin B\cos A = \sin(A-B)
\end{align*}

\begin{multline}
=\frac{ab}{2 \sin\theta\cos\theta}~\mydet{
 {\sin\theta} & {\cos\theta} & 1  \\ $-$
 \frac{\sin(\theta-\phi)}{\cos\phi} & \frac{\sin(\theta-\phi)}{\sin\phi} & 0  \\
 \frac{\sin(\gamma-\theta)}{\cos\gamma} &$-$ \frac{\sin(\gamma-\theta)}{\sin\gamma} & 0 
}
\end{multline}

$\because$~taking~ common ~$ C1 \xleftrightarrow{{\frac{1}{\cos\theta}}} C1$,~$C2 \xleftrightarrow{{\frac{1}{\sin\theta}}} C2$

\begin{multline}
=\frac{ab~(-\sin(\theta-\phi))~(-\sin(\gamma-\theta))}{2 \sin\theta\cos\theta}\\~\begin{vmatrix}
 {\sin\theta} & {\cos\theta} & 1  \\ 
 \frac{1}{\cos\phi} & \frac{1}{\sin\phi} & 0  \\
 \frac{1}{\cos\gamma} & \frac{1}{\sin\gamma} & 0 
\end{vmatrix}    
\end{multline}

$\because$~taking~ common ~$ R2 \xleftrightarrow{{\sin(\theta-\phi)}} R2$,\\$R3 \xleftrightarrow{{\sin(\gamma-\theta)}} R3$



\begin{multline}
=\frac{ab~\sin(\theta-\phi)~\sin(\gamma-\theta)}{ \sin2\theta}\\\left[\frac{1}{\cos\phi\sin\gamma}-\frac{1}{\sin\phi\cos\gamma}
\right]
\end{multline}

$\because$ solving~determinate 


\begin{multline}
=\frac{ab~\sin(\theta-\phi)~\sin(\gamma-\theta)}{ \sin2\theta}\\\left[\frac{\sin\phi\cos\gamma-\cos\phi\sin\gamma}{\sin\phi\cos\phi\sin\gamma\cos\gamma}
\right]
\end{multline}

$\because$ solving ~above~equation 

\begin{multline}
=\frac{ab~\sin(\theta-\phi)~\sin(\gamma-\theta)}{ \sin2\theta}\\\left[\frac{4~\sin(\phi-\gamma)}{4~\sin\phi\cos\phi\sin\gamma\cos\gamma}
\right]
\end{multline}


$\because \sin A\cos B - \sin B\cos A=\sin(A-B)$
\begin{align*}
~multiply ~and ~divide ~by ~4
\end{align*}

\begin{multline}
=\frac{4ab~\sin(\theta-\phi)~\sin(\gamma-\theta)~\sin(\phi-\gamma)}{ \sin2\theta~2\sin\phi\cos\phi~2\sin\gamma\cos\gamma}
\end{multline}

\begin{multline}
=\frac{4ab~\sin(\theta-\phi)~\sin(\gamma-\theta)~\sin(\phi-\gamma)}{ \sin2\theta~\sin2\phi~\sin2\gamma} \\= R.H.S
\end{multline}
\\\\
$Putting ~Some ~Numerical~ Values$
\begin{multline}
\frac{4~(1~1)~\sin(60-45)~\sin(30-60)~\sin(45-30)}{ \sin2(60)~\sin2(45)~\sin2(30)}\\
\end{multline}
putting~ value~ of~a=1,~b=1 $~\theta=60~\phi=45 \\ \gamma=30$

\begin{multline}
=\frac{-0.133974}{0.75}=-0.178632\\
\end{multline}

\end{document}